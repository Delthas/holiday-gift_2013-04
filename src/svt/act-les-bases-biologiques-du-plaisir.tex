\documentclass[12pt]{article}

\usepackage[utf8]{inputenc}
\usepackage[T1]{fontenc}
\usepackage[francais]{babel}

\renewcommand{\emph}{\textbf}

\title{\textbf{Les bases biologiques du plaisir}}
\date{}
\begin{document}

\maketitle

\section*{Document 1}

L'orgasme est un réflexe : il est dû à des phénomènes biologiques naturels et est accompagné de manifestations physiques involontaires.

L'orgasme est aussi psychologique : il êst accompagné de pensées et d'émotions variables selon l'individu, dépendant de son histoire personnelle et du contexte.

\section*{Document 2}

L'aire tegmentale ventrale est liée au plaisir car elle est stimulée en premier lors de l'orgasme.

Le \textit{circuit} est composé de différentes aires cérébrales qui seront activées lors d'un orgasme et la \textit{récompense} est le plaisir sexuel.

\section*{Visionnage de la vidéo}

La stimulation de l'ATV libère de la dopamine das le cerveau $\Rightarrow$ sensation de satisfaction.

Hypothalamus : zone finale du circuit.

\textit{Cf. schéma.}

\section*{Bilan}

L'activité sexuelle est essociée au plaisir qui repose entre autres sur des phénomènes biologiques.

Il s'agit de l'activation dans le cerveau d'un circuit particulier : le \emph{système de la récompense}.

Ce système est à l'origine du \emph{renforcement} de certains comportements (satisfaction $\Rightarrow$ motivation).

\end{document}