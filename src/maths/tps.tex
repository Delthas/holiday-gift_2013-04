\documentclass[12pt]{article}

\usepackage[utf8]{inputenc}
\usepackage[T1]{fontenc}
\usepackage[francais]{babel}

\usepackage{amsmath}
\usepackage{amssymb}
\usepackage{mathrsfs}

\renewcommand{\emph}{\textbf}

\newcommand{\degree}{\ensuremath{^\circ}}

\title{\textbf{Applications avec le produit scalaire}}
\date{}
\begin{document}

\maketitle

\section*{TP \#1}

\paragraph{A)}
\subparagraph{1.}
On a :
\begin{align*}
AB = 3\\
AC = 5\\
BC = 6
\end{align*}

D'où :
\begin{align*}
\overrightarrow{AB}.\overrightarrow{AC} &= \frac{1}{2}(\|\overrightarrow{AB}\|^2+\|\overrightarrow{AC}\|^2-\|\overrightarrow{AB}-\overrightarrow{AC}\|^2)\\
&= \frac{1}{2}(9 + 25 - 36)\\
&= -1
\end{align*}

\subparagraph{2.}

\begin{align*}
\overrightarrow{AB}.\overrightarrow{AC} &= AB \times AC \times \cos \widehat{BAC}\\
&= -1\\
\cos \widehat{BAC} &= - \frac{1}{AB \times AC}\\
&= - \frac{1}{15}
\end{align*}

Donc $\widehat{BAC} \approx 94\degree$.

\begin{align*}
\overrightarrow{BA}.\overrightarrow{BC} &= BA \times BC \times \cos \widehat{ABC}\\
&= \frac{1}{2} \times (BA^2 + BC^2 - CA^2)\\
&= \frac{1}{2} \times (9 + 36 - 25)\\
&= 10
\end{align*}

\begin{align*}
\cos \widehat{ABC} &= \frac{\overrightarrow{BA}.\overrightarrow{BC}}{BA \times BC}\\
&= \frac{10}{18} = \frac{5}{9}
\end{align*}

Donc $\widehat{ABC} \approx 56\degree$.

\begin{align*}
180 - (94 + 56) = \widehat{ACB}\\
\Leftrightarrow \widehat{ACB} = 30\degree
\end{align*}

\paragraph{B)}
\subparagraph{1.}
\begin{align*}
\overrightarrow{AB}.\overrightarrow{AC} &= AB \times AC \times \cos \widehat{BAC}\\
&= 35 \times \cos 65\degree\\
&\approx 14.8
\end{align*}

\begin{align*}
\overrightarrow{BC} &= \overrightarrow{BA} + \overrightarrow{AC} = \overrightarrow{AC} - \overrightarrow{AB}\\
&= \overrightarrow{BC}.\overrightarrow{BC}\\
&= (\overrightarrow{AC} - \overrightarrow{AB}).(\overrightarrow{AC} - \overrightarrow{AB})\\
&= \overrightarrow{AC}.\overrightarrow{AC} - \overrightarrow{AC}.\overrightarrow{AB} - \overrightarrow{AB}.\overrightarrow{AC} + \overrightarrow{AB}.\overrightarrow{AB}\\
&= \overrightarrow{AC}^2 - 2\overrightarrow{AC}.\overrightarrow{AB} + \overrightarrow{AB}^2
\end{align*}

D'où :
\begin{align*}
BC^2 &= 25 - 70 \times \cos 65\degree + 49\\
&= 74 - 70 \cos 65\degree
\end{align*}

\subparagraph{2.}

\section*{TP \#2}

\paragraph{1.}
\begin{align*}
AB &= \sqrt{(-2 - 2)^2 + (1 + 3)^2}\\
&= \sqrt{16 + 16}\\
&= \sqrt{32}\\
&= 4\sqrt{2}
\end{align*}

\begin{align*}
AC &= \sqrt{(3 - 2)^2 + (4 + 3)^2}\\
&= \sqrt{1 + 49}\\
&= \sqrt{50}\\
&= 5\sqrt{2}
\end{align*}

\begin{align*}
\overrightarrow{AB}.\overrightarrow{AC} &= -4 \times 1 + 7 \times 4\\
&= 24
\end{align*}

\paragraph{2.}
\begin{align*}
\cos \widehat{BAC} &= \frac{\overrightarrow{AB}.\overrightarrow{AC}}{AB \times AC}\\
&= \frac{24}{4\sqrt{2} \times 5 \times \sqrt{2}}\\
&= \frac{24}{40}\\
&= 0.6
\end{align*}
\[
\widehat{BAC} \approx 0.9 rad
\]

\paragraph{3.}
Correction dans l'énoncé :
\[
\overrightarrow{AB}.\overrightarrow{AC} = \overrightarrow{AB}.\overrightarrow{AH} = 24
\]

On pose $H(x_H;y_H)$.

On sait que :
\[
\overrightarrow{AB}(-4;4)\\
\overrightarrow{AC}(1;7)
\]

On a :
\[
\overrightarrow{AH}(x_H - 2;y_H +3)\\
\]

\begin{align*}
&\overrightarrow{AB}.\overrightarrow{AH} = -4 \times (x_H - 2) + 4 \times (y_H + 3) = 24\\
&\Leftrightarrow -4x_H + 8 + 4y_H + 12 = 24\\
&\Leftrightarrow -4x_H + 4y_H + 20 - 24 = 0\\
&\Leftrightarrow -4x_H + 4y_H - 4 = 0\\
&\Leftrightarrow 4(-x_H + y_H - 1) = 0\\
&\Leftrightarrow -x_H + y_H - 1 = 0
\end{align*}

Déterminer une équation de (AB).

$\overrightarrow{AB}(-4;4)$ est un vecteur directeur de la droite $(AB)$.

Soit $\overrightarrow{u}(-1;1)$. $\overrightarrow{u}$ est donc colinéaire à $\overrightarrow{AB}$ et vecteur directeur de $(AB)$.

\begin{align*}
M(x;y) \in (AB) &\Leftrightarrow \overrightarrow{AM} \text{et} \overrightarrow{u} \text{colinéaires}\\
&\Leftrightarrow (x - 2) \times (1) - (y + 3) \times (-1) = 0\\
&\Leftrightarrow x - 2 + y + 3 = 0\\
&\Leftrightarrow x + y + 1 = 0
\end{align*}

\[
H(x_H; y_H) \in (AB)
\]

\[
\left\{
\begin{array}{r c l}
-x_H + y_H - 1 = 0\\
x_H + y_H + 1 = 0
\end{array}
\right.
\]

\begin{align*}
2y_H = 0\\
\Leftrightarrow y_H = 0
\end{align*}

\begin{align*}
-x_H - 1 = 0\\
\Leftrightarrow x_H = -1
\end{align*}

D'où :

\begin{align*}
H(-1;0)
\end{align*}

On peut alors déterminer l'équation de $(CH)$ :

\begin{align*}
y &= x + 1
\end{align*}

\end{document}